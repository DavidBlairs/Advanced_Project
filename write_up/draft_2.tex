\documentclass[a4paper,12pt]{book}
\usepackage{amsmath}
\usepackage{bm}
\usepackage{amsfonts}
\usepackage{setspace}
\usepackage{helvet}
\usepackage{graphicx}
\usepackage[style = numeric, backend = bibtex]{biblatex}
\graphicspath{ {./images/} }

\newcommand{\set}[1]{\mathcal{#1}}
\newcommand{\operation}{\bigotimes}
\newcommand{\matrx}[1]{\bm{#1}}
\newcommand{\vectr}[1]{\textbf{#1}}
\newcommand{\real}{\mathbb{R}}
\newcommand{\integers}{\mathbb{Z}}
\newcommand{\italic}[1]{\textit{#1}}

% new environments 
% recto  inverso 
% set book class to one sided
% add chapter levels
% new environment / draft notes / if then else draft = 1 / check the internet for prior examples

\doublespacing

\newtheorem{definition}{Definition}[section]

\addbibresource{sources/sources.bib}

\begin{document}
	\section{Vector Space}
	% vector space introduction
	A vector space is an algebraic structure that consists of a set of vectors and two operations defined on these vectors: vector addition and scalar multiplication. The elements of the vector space are called vectors, which can be added together and multiplied by numbers, called scalars. The operations of vector addition and scalar multiplication must satisfy a set of axioms, which ensure that the resulting vectors remain within the vector space. It will be helpful to describe a vector space by first explaining the concept of a set and a group. 
	
	% set definition
	A set is a collection of mathematical objects such as numbers, lines or potentially other sets. They were first formalised by George Cantor in 
	1895 \cite{cantor1895beitrage} as being either infinite or finite and containing distinct elements. 
	% group definition
	\begin{definition}
		\normalfont
		Let $ \mathcal{V} $ be a set and $ \operation $ denote a binary operation between elements of this set such that $ \operation : \set{G} \times \set{G} \rightarrow \set{G} $. $G := (\set{G}, \operation)$ is called a group  \cite{mml_group_36} if the following conditions are met and specific elements are present:
		\begin{enumerate}
			\item (Closure) The binary operation between any two elements of the set will result in an element which is also part of the set: $ \operation : \forall x,y \in \set{G} : x \operation y \in \set{G} $.
			\item (Associativity) The way in which elements of the set are combined within a larger expression does not affect the result: $ \forall x, y, z \in \set{G} : (x \operation y) \operation = x \operation (y \operation z) $.
			\item (Neutral Element) A neutral element in a set is an element that, when combined with any other element in the set using the group operation, results in the same element: $ \exists e  \in \set{G} \forall x \in \set{G} : x \operation e = x$ and $ e \operation x = x $.
			\item (Inverse Element) The inverse of an element in a set is an element that when combined with the original element using the group operation, results in the neutral element. It allows for the reversal of the group operation: $ \forall x \in \set{G} \exists y \in \set{G} :  x \operation y = e$ and $ y \operation x = e $, where $ e $ is the neutral element. We can denote the inverse of an element $ x $ as $ x^{-1} $.
		\end{enumerate}
	\end{definition}
	
	% relevence of groups 
	Groups are very important in many areas of mathematics. Their rigorous and formal definition means that, if something is found to be a group, its properties can be better understood in the specific context in which it is relevant.
	% example of group 
	% ammend: general linear group being in italic / reference from the book / regular (invertible) not regular invertible
	An example of a group is the general linear group. This is the set of regular invertible matrices $ \matrx{A} \in \real^{n \times n} $ with respect to matrix multiplication.
	% definition of an abelian group
	Given a group $ G $, if the order in which the group operation is performed does not matter i.e. $ \forall x, y \in \set{G} : x \operation y = y \operation x $, then $ G = (\set{G}, \operation) $ is known as an \italic{Abelian group} (commutative). An example of this would be $ (\integers, +) $, the set of all integers under the addition operation.
	
	% definition of a vector space
	A vector space \cite{mml_vector_space_37} is a special type of group with some additional conditions. 
	
	\begin{definition}[Vector Space]
		\normalfont A real-valued $\italic{vector space}$ $V = (\set{V}, +, \cdot)$ is a set $\set{V}$ with two operations:
		\begin{align}
			+: \set{V} &\times \set{V} \rightarrow \set{V} \hspace{10pt} (\text{Inner Operation}) \\
			\cdot: \real &\times \set{V} \rightarrow \set{V} \hspace{10pt} (\text{Outer Operation})
		\end{align}
		where:
		\begin{enumerate}
			\item $(\set{V}, +)$ is an Abelian group
			\item (Distributivity) The outer operation can be ``Distributed'' across elements either before or after the inner operation has occurred: 
			\begin{enumerate}
				\item $\forall \lambda \in \real, \vectr{x}, \vectr{y} \in \mathcal{V}: \lambda \cdot \vectr{x} + \lambda \cdot \vectr{y}$
				\item $\forall \lambda, \psi \in \real, \vectr{x} \in \set{V}: (\lambda + \psi) \cdot \vectr{x} = \lambda \cdot \vectr{x} + \psi \cdot \vectr{x}$
			\end{enumerate} 
			\item Associativity (outer operation): $\forall \lambda, \psi \in \real, \vectr{x} \in \set{V}: \lambda \cdot (\psi \cdot \vectr{x}) = (\lambda \psi) \cdot \vectr{x}$
			\item Neutral element with respect to the outer operation: $\forall \vectr{x} \in \set{V}: 1 \cdot \vectr{x} = \vectr{x}$
		\end{enumerate}
	\end{definition}
	
	% generalisation of a vector
	We commonly think of vectors as being mathematical objects with both direction and magnitude. However, these are only geometric vectors. More generally, any set of objects which follows the definition of a vector space is known as a vector. For example, polynomials are also vectors. Two can be added together, resulting in another polynomial and they can be multiplied by a scalar $ \lambda \in \real $ which again results in another polynomial. 
	
	% ammend: add some more details here about how the specific examples fit into the axioms of a vector space. You should also include an example of a finite dimensional and infinite dimensional situation.
	
	
	% begin linear independence
	\section{Linear Independence}
	Linear independence \cite{mml_linear_combinations_40} is a property of a set of vectors which describes whether there is any redundancy with respect to the linear combinations of these vectors. 
	% definition of linear combination
	\begin{definition}[Linear Combination]
		\normalfont	Consider a vector space $\italic{V}$ and a finite number of vectors $\vectr{x}_i,\ldots,\vectr{x}_k \in \italic{V}$. Then, every $\vectr{v} \in \italic{V}$ of the form
		\begin{align}
			\vectr{v} = \lambda_1\vectr{x}_1 + \cdots + \lambda_k\vectr{x}_k =  \sum_{i=1}^{k} \lambda_i\vectr{x}_i \in \italic{V}
		\end{align}
		with $\lambda_1,\ldots,\lambda_k \in \real$ is a $\italic{linear combination}$ of the vectors $\vectr{x}_1,\cdots,\vectr{x}_k$.
	\end{definition}
	% base vectors of R^2
	Consider two vectors in $ \real^2 $, $\vectr{e}_1 = (1, 0)^{\top}$ and $\vectr{e}_2 = (0, 1)^{\top}$. It is common to see these written as $ \vectr{i} $ and $ \vectr{j} $ respectively. We can represent any vector in $ \real^2 $ as a linear combination of these two vectors. 
	% definition of linear independence
	\begin{definition}[Linear (In)dependence] 
		\normalfont Let us consider a vector space $\italic{V}$ with $k \in \real$ and  $\vectr{x}_1,\cdots,\vectr{x}_k \in \italic{V}$. If there is a non-trivial combination, such that $\vectr{0} = \sum_{i=1}^{k} \lambda_i\vectr{x}_i$ with at least one $\lambda_i \ne 0$, the vectors  $\vectr{x}_1,\cdots,\vectr{x}_k$ are \italic{linearly dependent}. If only the trivial solution exists, i.e., $ \lambda_1 = \cdots = \lambda_k = 0 $ the vectors $\vectr{x}_1,\cdots,\vectr{x}_k $ are \italic{linearly independent}.
		\label{def:linear_independence}
	\end{definition}
	% example of linear independence through linear combinations
	Lets consider the following set $\set{V}$ of vectors $\vectr{v}_1,\vectr{v}_2 $ and $ \vectr{v}_3$ where (respectively):
	\begin{align}
		\set{V} = \left\{\begin{pmatrix} 1 \\ 2 \end{pmatrix},\begin{pmatrix} -2 \\ -4 \end{pmatrix},\begin{pmatrix} -4 \\ -1 \end{pmatrix} \right\}
	\end{align}
	And let $\vectr{y} \in \real^2$ be any linear combination that can be made from these vectors. Writing this out explicitly with $\alpha, \beta, \omega \in \real$:
	\begin{align}
		\vectr{y} &= \alpha \begin{pmatrix} 1 \\ 2 \end{pmatrix} + \beta \begin{pmatrix} -2 \\ -4 \end{pmatrix} + \omega \begin{pmatrix} -4 \\ -1 \end{pmatrix} \\
		&= \alpha \begin{pmatrix} 1 \\ 2 \end{pmatrix} - 2\beta \begin{pmatrix} 1 \\ 2 \end{pmatrix} + \omega \begin{pmatrix} -4 \\ -1 \end{pmatrix} \\
		&= (\alpha - 2\beta) \begin{pmatrix} 1 \\ 2 \end{pmatrix} + \omega \begin{pmatrix} -4 \\ -1 \end{pmatrix} \\
	\end{align} 
	% intuition behind linear independence 
	What we have shown is that if a vector $ \vectr{y} $ can be represented as a linear combination of $ \vectr{v}_1,\vectr{v}_2 $ and $ \vectr{v}_3 $ then we can represent $ \vectr{y} $ as a linear combination of $ \vectr{v}_2 $ and $ \vectr{v}_3 $. This is because $ \vectr{v}_1 $ is a scaled version of $ \vectr{v}_2 $ and vice versa. 
	% leading into the formal definition 
	More formally:
	
	\begin{align}
		\begin{pmatrix} -2 \\ -4 \end{pmatrix} + 2 \begin{pmatrix} 1 \\ 2 \end{pmatrix} = \vectr{0}
	\end{align}
	
	which following from definition \ref{def:linear_independence} demonstrates that we have a linearly dependent set of vectors.
	
	\section{Generating Set and Span}
	
	A generating set and the span of a set of vectors are concerned with the vector space produced by the linear combination of all the vectors in the set.
	
	% definition of a generating set and span
	\begin{definition}[Generating Set and Span]
		\normalfont Consider a vector space $\textit{V} = (\mathcal{V}, +, \cdot)$ and set of vectors $\mathcal{A} = \{\boldsymbol{x}_1,\ldots,\boldsymbol{x}_k\} \subseteq \textit{V}$. If every vector $\boldsymbol{v} \in \mathit{V}$ can be expressed as a linear combination of vectors of $\boldsymbol{x}_1,\ldots,\boldsymbol{x}_k$, $\mathcal{A}$ is called the generating set of $\textit{V}$. The set of all linear combinations of vectors in $\mathcal{A}$ is called the span of $\mathcal{A}$. If $\mathcal{A}$ spans the vector space $\textit{V}$, we write $\textit{V} = span[\mathcal{A}]$. 
	\end{definition}

	Consider the vector space $ V = (\real, +, \cdot) $ under matrix addition and scalar multiplication 

	\printbibliography
	
\end{document}